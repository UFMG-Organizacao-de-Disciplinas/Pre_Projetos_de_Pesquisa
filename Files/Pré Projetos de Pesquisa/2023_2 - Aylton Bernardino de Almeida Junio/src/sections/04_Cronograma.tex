\section{Cronograma}
\label{sec:schedule}

Nesta seção é proposto um cronograma para realização das atividades mencionadas, conforme apresentado na Tabela \ref{tab:schedule}. Ele prevê um período de dois anos para realização do projeto, tendo seu início no segundo semestre de 2023 e seu fim no primeiro semestre de 2025.

\begin{table}[!htb]
  \caption{Cronograma de atividades proposto.}
  \label{tab:schedule}
  \resizebox{\textwidth}{!}{%
    \begin{tabular}{|l|c|c|c|c|}
      \hline
      ~                                                            & \textbf{2023}        & \multicolumn{2}{c}{\textbf{2024}} \vline & \textbf{2025}                               \\ \hline
      \textbf{Tarefas}                                             & \textbf{2º Semestre} & \textbf{1º Semestre}                     & \textbf{2º Semestre} & \textbf{1º Semestre} \\ \hline
      Projeto e Análise de Algoritmos                              & X                    & ~                                        & ~                    & ~                    \\ \hline
      Engenharia de Software                                       & X                    & ~                                        & ~                    & ~                    \\ \hline
      Inteligência Artificial                                      & ~                    & X                                        & ~                    & ~                    \\ \hline
      Análise e Modelagem de Desempenho de Sistemas de Computação  & ~                    & X                                        & ~                    & ~                    \\ \hline
      Fundamentos Teóricos da Computação                           & ~                    & ~                                        & X                    & ~                    \\ \hline
      Tópicos em Engenharia de Software ou Inteligência Artificial & ~                    & ~                                        & X                    & ~                    \\ \hline
      Primeiro Artigo Científico                                   & ~                    & X                                        & X                    & ~                    \\ \hline
      Segundo Artigo Científico                                    & ~                    & ~                                        & X                    & X                    \\ \hline
      Dissertação                                                  & ~                    & X                                        & X                    & X                    \\ \hline
    \end{tabular}
  }
\end{table}


O \textbf{2º semestre de 2023} teria como principal objetivo a realização das disciplinas \textbf{Projeto e Análise de Algoritmos} e \textbf{Engenharia de Software}. Ambas disciplinas são consideradas fundamentais para o desenvolvimento do projeto, de forma que iniciar por elas é de grande importância. Além disso, também será neste semestre que o aluno proponente terá que se familiarizar com o curso de Mestrado e com a área de pesquisa.

Durante o \textbf{1º semestre de 2024} é proposto que a disciplina \textbf{Inteligência Artificial} seja cursada. Ela é considerada fundamental para o projeto, sendo importante cursá-la logo que possível. Também é previsto que a disciplina \textbf{Análise e Modelagem de Desempenho de Sistemas de Computação} sejam cursadas, de forma a melhor capacitar o candidato a definir métricas e experimentos que sejam congruentes com a realidade do desenvolvimento de \textit{software}. Neste mesmo semestre é esperado que sejam iniciados ambos o \textbf{primeiro artigo científico} proposto e a \textbf{dissertação} do mestrado.

A disciplina \textbf{Fundamentos Teóricos da Computação} será cursada no \textbf{2º semestre de 2024}. Também é proposto que a disciplina \textbf{Tópicos em Engenharia de Software} ou \textbf{Tópicos em Inteligência Artificial} seja cursada nesse semestre. Elas são consideradas importantes para o projeto, porém não fundamentais, de forma que podem ser cursadas em um momento posterior. Durante este semestre também é proposto que o \textbf{primeiro artigo científico} seja finalizado e que o \textbf{segundo artigo científico} iniciado, sendo definido seu tema e principais objetivos.

No \textbf{1º semestre de 2025} é proposto que o \textbf{segundo artigo científico} proposto seja finalizado e publicado. Durante este período também é previsto que a dissertação seja finalizada a fim de apresentar os resultados obtidos durante o curso e propor trabalhos futuros.
