\section{Introdução}
\label{sec:intro}

O estudo sobre inteligência artificial (IA) e aprendizado de máquina não é algo recente, tendo sido tópico de discussão na sociedade desde meados do século XX.
Porém, o lançamento de ferramentas como o \textit{Chat GPT-4} e \textit{Google Bard} para o público geral vem mostrando a capacidade que essa tecnologia tem de alterar como tarefas são executadas nas mais diversas áreas. Ambas aplicações são \textit{chatbots} que utilizam de IA para responder perguntas simulando linguagem natural \cite{openai-chatgpt,google-bard}, e demonstraram a capacidade de gerar respostas plausíveis para diversos tópicos. Alguns exemplos são a escrita de textos bem estruturados como \textit{posts} para um \textit{blog}, e necessidades mais abstratas, como responder perguntas relacionadas à filosofia e à sociologia.

Enquanto a utilidade dessas ferramentas para o público geral ainda está sendo descoberta, a utilização de tecnologias de IA para o desenvolvimento de \textit{software} não é uma novidade. Modelos como o \textit{Codex} da \textit{Open AI}, capazes de analisar linguagem natural a fim de escrever blocos de código em resposta \cite{openai-codex}, vem sendo utilizados há alguns anos na tentativa de otimizar o desenvolvimento de sistemas. Dentro desse caso de uso, esses modelos têm sido capazes de auxiliar desenvolvedores a encontrar possíveis \textit{bugs} por meio da análise de algoritmos previamente escritos \cite{codex-bugfix}. Demonstram também, a capacidade de gerar trechos de código por meio de ferramentas como o \textit{Github Copilot} e o \textit{Amazon CodeWhisperer}, ambas utilizadas para gerar sugestões em um editor de texto por meio da utilização de inteligência artificial \cite{github-copilot,code-whisperer}.

Ainda que ferramentas de IA tenham se mostrado úteis para o desenvolvimento de \textit{software}, há muito o que se estudar quanto às diversas formas em que elas podem ser aplicadas para melhor atender desenvolvedores. Tendo isso em vista, neste projeto é proposto um trabalho na área de \textbf{Engenharia de Software}, que tem como objetivo \textbf{avaliar a capacidade de ferramentas que utilizam de inteligência artificial para aprimorar o processo de manutenção e evolução de sistemas de \textit{software}.} Dessa mesma forma, três objetivos específicos são definidos, sendo eles:
\begin{enumerate}
  \item Verificar se existem melhorias de desempenho notáveis ao comparar desenvolvedores que utilizam ferramentas de IA durante o desenvolvimento em relação aos que não utilizam;
  \item Analisar se existe algum impacto na qualidade do código resultante da utilização de ferramentas de IA;
  \item Estudar possíveis formas de utilizar IAs a fim de tornar a manutenção de sistemas mais eficiente.
\end{enumerate}

Este documento está organizado da seguinte forma: A Seção \ref{sec:theory} busca apresentar o referencial teórico que servirá de fundamentação para o projeto. Na Seção \ref{sec:methodology}, é abordada a metodologia proposta para realização do projeto. Por fim, a Seção \ref{sec:schedule} descreve o cronograma de atividades proposto.
