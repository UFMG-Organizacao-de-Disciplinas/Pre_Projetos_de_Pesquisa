\section{Metodologia}
\label{sec:methodology}

O projeto proposto é classificado como exploratório, tendo como objetivo a realização de estudos e experimentos a serem seguidos a fim de melhor explorar o tema proposto. Nesta seção, são descritas as atividades para a realização do projeto, assim como as etapas necessárias para a conclusão de cada uma delas.

\subsection{Atividades}

Tendo como referência os três objetivos específicos propostos, se faz necessário definir quais as perguntas que devem ser respondidas a fim de alcançar esses objetivos. Para isso, um estudo maior precisará ser feito com o objetivo de expandir o referencial teórico apresentado na Seção \ref{sec:theory} para propor perguntas que auxiliem na evolução do conhecimento já existente na área. Tendo definido os questionamentos, um estudo mais voltado para métricas deverá ser conduzido a fim de definir quais serão utilizadas durante os experimentos para avaliar os resultados obtidos. Por fim, será necessário definir quais serão os experimentos a serem conduzidos, assim como a maneira pela qual eles serão realizados.

\subsection{Disciplinas}

Ao longo do projeto, algumas disciplinas serão cursadas a fim de melhor capacitar o candidato a conduzir os experimentos propostos. Elas podem ser divididas em dois grupos, o primeiro sendo de disciplinas fundamentais e que servirão de base para o projeto. Já o segundo grupo é composto de disciplinas complementares que servirão de auxílio para a realização do trabalho proposto, mas que poderão ser substituídas por outras disciplinas conforme a necessidade.

Tendo isso em vista, a primeira disciplina do primeiro grupo é a \textbf{Projeto e Análise de Algoritmos}. Esta é a única matéria obrigatória do curso de mestrado. Ela será responsável por capacitar o aluno a analisar a complexidade de algoritmos e a propor soluções para problemas computacionais, servindo de base para os demais estudos propostos. A segunda disciplina proposta é a chamada \textbf{Engenharia de Software}. A engenharia de \textit{software} é parte fundamental da manutenção e evolução de sistemas, de forma que seu estudo será de grande importância para melhor compreender este processo e definir experimentos congruentes com sua realidade. Por fim, também é necessário cursar a disciplina de \textbf{Inteligência Artificial}. Essa visa a abordar fundamentos da inteligência artificial e mostrar suas aplicações em problemas práticos. Ela será de grande importância para maior entendimento de como essa tecnologia funciona, a fim de melhor definir e analisar experimentos.

Como parte do segundo grupo de disciplinas serão mencionadas apenas três, porém, como mencionado previamente, elas poderão ser alteradas para melhor anteder às necessidades do projeto. Dentre elas, a primeira é a \textbf{Análise e Modelagem de Desempenho de Sistemas de Computação}. Essa tem como objetivo apresentar técnicas de análise e modelagem de desempenho de sistemas de computação. Ela será de grande importância para a definição de métricas a serem utilizadas durante os experimentos. A segunda disciplina é a \textbf{Fundamentos teóricos da Computação}. Essa procura apresentar os fundamentos matemáticos nos quais se baseia a computação, sendo de grande importância para o estudo sobre o funcionamento de IA. Por fim, há a possibilidade de cursar a disciplina de \textbf{Tópicos em Engenharia de Software} ou \textbf{Tópicos em Inteligência Artificial}. Ambas matérias se relacionam com o tema proposto, entretanto, como elas possuem uma ementa variável, seria interessante avaliar qual das duas cursar com base no tópico discutido no semestre e na sua capacidade de melhor qualificar o candidato para execução do projeto.
